\documentclass[12pt]{report}

\usepackage{palatino}
\usepackage{hyperref}
\usepackage[margin=1in]{geometry}
\renewcommand{\baselinestretch}{1.5}

\begin{document}

\title{Alshayun: A mobile education application}
\author{Jacob Chappell}
\date{\today}
\maketitle

\tableofcontents

% Definitions in bold: \textbf{}

\chapter{Introduction}

    \section{What is Alshayun?}

\textbf{Alshayun} is a mobile application for delivering articles consisting of
rich text and interactive \textbf{applet}s to readers. Written in the portable
\textbf{Ionic} framework, Alshayun is capable of running on the Web, Android
devices, and Apple iOS devices. However, Alshayun has currently only been tested
on a Samsung Galaxy S7 device running Android.

Alshayun is designed with three actors in mind: the \textbf{content author}, the
\textbf{reader}, and the \textbf{developer}. The content author is anyone who
has anything they would like to write about, potentially making use of
interactive and embedded applets as assistive visual aids. The reader is anyone
who would like to read what one or more content authors have to say. The
developer is the one capable of developing applets and other functionality of
Alshayun of which content authors and readers can make use. Note that while I
speak of these roles in the singular, many individuals may inhabit any given
role, and some individuals may inhabit multiple roles.

    \section{Inspiration}

After taking a numerical methods course as part of my Computer Science
undergraduate degree, I became fascinated with Bézier curves. Whilst researching
Bézier curves online, I came across an article titled \textit{A Primer on Bézier
Curves} \cite{pomax} authored by someone named Pomax. The highly detailed and
enlightening article was filled with interactive applets designed to strengthen
the reader's understanding of each progressively difficult concept.

Intrigued by Pomax's work, I was inspired to develop a sort of content delivery
mechanism designed to allow content authors like Pomax author similar articles
and deliver them to interested readers. Supporting the development and use of
interesting, interactive applets was a priority. I recognized that any such
modern application needed to be mobile-friendly, and so I decided to develop
exclusively with mobility in mind.

    \section{About the Name}

Alshayun was originally meant to be tailored towards mathematics education.
However, during the process of developing the application, I realized that I had
created a much more generalizable platform capable of servicing any subject and
perhaps more. Albeit, I decided to keep the name I gave the application from the
time of its inception---a name which carries with it certain mathematical
connotations.

Algebra (from the Arabic ``al-jabr'') is one of the most powerful mathematical
devices and fields of study, as well as my personal favorite subject of
mathematics. The English-speaking world inherited its written algebraic works
from Latin, which came from Spanish, which came from Arabic. This line of
inheritance spanned many years and involved merchants and trade routes, among
other things. In the original Arabic sources, the word ``al-shayun,'' meaning
``the unknown thing,'' was frequently used to describe the unknown in algebraic
equations. Therefore, I found ``al-shayun'' to be an appropriate name filled
with mathematical historical meaning.

To learn more about how ``al-shayun'' came to be the letter ``x'' used in modern
algebra, check out the article \textit{Why X marks the unknown} by Terry Moore
\cite{moore}.

\chapter{Technologies Used}
    \section{Node.js}
    \section{Angular}
    \section{Ionic}
    \section{Cordova}
    \section{Flask}
    \section{Singularity}

\chapter{Quick n' Dirty Server}
    \section{Building and Running}
    \section{Motivation}
    \section{Backend}
    \section{Frontend}
    \section{Production Considerations}

\chapter{Alshayun Mobile Application}
    \section{Building and Running}
    \section{User Interface}
    \section{Caching Strategy}
    \section{Applets}
    \section{Accepting Contributions}

\chapter{Use Cases}
    \section{Classroom Auxiliary Content}
    \section{Starter Mobile Blog}

\chapter{Future Work}
    \section{User Accounts}
    \section{Cloud Hosting}
    \section{Power Efficiency}

\chapter{Conclusion}
    \section{Learning Outcomes}

\begin{thebibliography}{9}
    \bibitem{moore} Moore, Terry. \textit{Why X marks the unknown.} Retrieved
        from \url{https://cosmosmagazine.com/mathematics/why-x-marks-unknown-0}.
    \bibitem{pomax} Pomax. \textit{A Primer on Bézier Curves.} Retrieved from
        \url{https://pomax.github.io/bezierinfo/}.
\end{thebibliography}

\end{document}
