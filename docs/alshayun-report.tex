\documentclass[12pt]{report}

\usepackage{palatino}
\usepackage{hyperref}
\usepackage[margin=1in]{geometry}
\renewcommand{\baselinestretch}{1.5}

\begin{document}

\title{Alshayun: A mobile education application}
\author{Jacob Chappell}
\date{\today}
\maketitle

\tableofcontents

% Definitions in bold: \textbf{}

\chapter{Introduction}

    \section{What is Alshayun?}

\textbf{Alshayun} is a mobile application for delivering articles consisting of
rich text and interactive \textbf{applet}s to readers. Written in the portable
\textbf{Ionic} framework, Alshayun is capable of running on the Web, Android
devices, and Apple iOS devices. However, Alshayun has currently only been tested
on a Samsung Galaxy S7 device running Android.

Alshayun is designed with three actors in mind: the \textbf{content author}, the
\textbf{reader}, and the \textbf{developer}. The content author is anyone who
has anything they would like to write about, potentially making use of
interactive and embedded applets as assistive visual aids. The reader is anyone
who would like to read what one or more content authors have to say. The
developer is the one capable of developing applets and other functionality of
Alshayun of which content authors and readers can make use. Note that while I
speak of these roles in the singular, many individuals may inhabit any given
role, and some individuals may inhabit multiple roles.

    \section{Inspiration}

After taking a numerical methods course as part of my Computer Science
undergraduate degree, I became fascinated with Bézier curves. Whilst researching
Bézier curves online, I came across an article titled \textit{A Primer on Bézier
Curves} \cite{pomax} authored by someone named Pomax. The highly detailed and
enlightening article was filled with interactive applets designed to strengthen
the reader's understanding of each progressively difficult concept.

Intrigued by Pomax's work, I was inspired to develop a sort of content delivery
mechanism designed to allow content authors like Pomax author similar articles
and deliver them to interested readers. Supporting the development and use of
interesting, interactive applets was a priority. I recognized that any such
modern application needed to be mobile-friendly, and so I decided to develop
exclusively with mobility in mind.

    \section{About the Name}

Alshayun was originally meant to be tailored towards mathematics education.
However, during the process of developing the application, I realized that I had
created a much more generalizable platform capable of servicing any subject and
perhaps more. Albeit, I decided to keep the name I gave the application from the
time of its inception---a name which carries with it certain mathematical
connotations.

Algebra (from the Arabic ``al-jabr'') is one of the most powerful mathematical
devices and fields of study, as well as my personal favorite subject of
mathematics. The English-speaking world inherited its written algebraic works
from Latin, which came from Spanish, which came from Arabic. This line of
inheritance spanned many years and involved merchants and trade routes, among
other things. In the original Arabic sources, the word ``al-shayun,'' meaning
``the unknown thing,'' was frequently used to describe the unknown in algebraic
equations. Therefore, I found ``al-shayun'' to be an appropriate name filled
with mathematical historical meaning.

To learn more about how ``al-shayun'' came to be the letter ``x'' used in modern
algebra, check out the article \textit{Why X marks the unknown} by Terry Moore
\cite{moore}.

\chapter{Technologies Used}

    \section{Node.js}

\textbf{Node\@.js} (Node) \cite{nodejs} is a JavaScript runtime designed mainly
to facilitate writing scalable, server-side software in JavaScript. Before Node,
first released in May of 2009, the idea of writing server-side software in
JavaScript was relatively unknown. Since its conception, Node has received much
attention and both praise and criticism from developers.

Less than a year after Node's inception, a package manager was released in
January of 2010 named the \textbf{Node Package Manager} (npm, intentionally not
capitalized) \cite{npm}. npm has since played a vital role in the development of
most JavaScript software and frameworks and has transcended its original intent
to become a general JavaScript package manager.

While developing Alshayun, I never interacted directly with Node. However, npm
was a vital part of my development process. It is for that reason that I mention
npm and Node as a used technology.

    \section{TypeScript}

\textbf{TypeScript} \cite{typescript} is a superset of JavaScript which
ultimately compiles into JavaScript. TypeScript complements JavaScript with type
safety and true object-oriented programming constructs such as classes, access
modifiers, and inheritance. In TypeScript, every variable and symbol in general
will have a type, either stated explicitly by the programmer or inferred by the
TypeScript compiler from usage. The TypeScript compiler is available for
installation as an npm package.

As the programming language of Alshayun, TypeScript has been extraordinarily
useful in development. In particular, the access to object-oriented programming
constructs has allowed me to write clean, modular, and purpose-driven code.

    \section{Angular}

\textbf{Angular} \cite{angular} is a Web application framework developed by
Google and written in TypeScript.  Relatively new, Angular was released in
September of 2016, although it's based on a rewrite of its older predecessor,
AngularJS. Angular is useful for developing responsive, single-page Web
applications backed by a full suite of development tools to assist in
application development, testing, and deployment. Angular is available for
installation as an npm package.

The single-page architecture of Angular is accomplished by the backing
Web-server rewriting URLs and redirecting requests to the root index\@.html file
of the Angular application. Angular then uses an internal router module to load
the appropriate page of the application. The application can be fully loaded up
front, or lazily loaded on demand. Lazy loading is recommended for larger
applications to avoid a lengthy initial page load.

One of Angular's strongest development points is its use of the
\textbf{Model-View-Controller} (MVC) design pattern. Angular provides a template
syntax for easily binding the model (data, variables) to the view (HTML elements
such as forms). Thus, if a variable \texttt{title} exists and the application
has bound it to the heading tag of a page with the syntax
\texttt{<h1>\{\{title\}\}</h1>}, then simply updating the \texttt{title}
variable will automatically result in the heading of the page being updated. The
developer need not worry about the details or be concerned with updating the
view his or herself. Furthermore, interactive actions such as button clicks can
trigger the calling of methods which further update the model.
    
    \section{Ionic}

\textbf{Ionic} \cite{ionic} is a framework for building cross-platform
applications in Hypertext Markup Language (HTML), Cascading Style Sheets (CSS),
and JavaScript. Ionic supports developing an application with a single codebase
which deploys to Web, Android, and iOS devices. As of version 4, the Ionic
codebase is divorced of an underlying JavaScript framework, although Angular is
still the most supported and widely used base for Ionic.

Ionic provides a collection of HTML tags and CSS which generate components
similar to \textbf{Bootstrap} \cite{bootstrap}. Examples of components include
buttons, cards, toasts, modals, and lists. Ionic components are designed mimic
the look and feel of a native Android or iOS application, complete with built-in
gestures and animations.
    
    \section{Cordova}

\textbf{Cordova} \cite{cordova} is an Apache project which provides a uniform
interface for generating device-dependent code for Android and iOS. For example,
Cordova provides a JavaScript interface for interacting with the built-in camera
of mobile devices. Thus, the developer can write one piece of code for taking
pictures and gathering image data, and Cordova automatically translates that
code into the appropriate programming language and format for desired devices
(Java for Android, Swift for iOS). Cordova is a vital part of Ionic, and Ionic
provides a direct command-line interface to Cordova for building mobile
packages.
    
    \section{Flask}

\textbf{Flask} \cite{flask} is a Python framework for the rapid development of
RESTful APIs (Representational State Transfer, Application Programming
Interface). Flask allows developers to prefix Python functions with decorators
indicating the URL endpoint which triggers the function, the acceptable HTTP
methods, and more. Flask also provides a collection of helpful methods for
generating HTTP responses, handling exceptions, and easily processing HTTP
request data. A built-in development server is provided for prototyping
purposes.
    
    \section{Singularity}

\textbf{Singularity} \cite{singularity} is a containerization software which
allows users to develop, package, and relocate full-fledged compute environments
consisting of an operating system and software binaries and libraries.
Singularity is one of several containerization platforms in existence. However,
it stands apart by being the most secure and, in my opinion, simplest to use. I
have a personal connection to Singularity, having contributed code to the
open-source project and developed many containers as part of my employment at
the University of Kentucky.

\chapter{Quick n' Dirty Server}
    \section{Building and Running}
    \section{Motivation}
    \section{Backend}
    \section{Frontend}
    \section{Production Considerations}

\chapter{Alshayun Mobile Application}
    \section{Building and Running}
    \section{User Interface}
    \section{Caching Strategy}
    \section{Applets}
    \section{Accepting Contributions}

\chapter{Use Cases}
    \section{Classroom Auxiliary Content}
    \section{Starter Mobile Blog}

\chapter{Future Work}
    \section{User Accounts}
    \section{Cloud Hosting}
    \section{Power Efficiency}

\chapter{Conclusion}
    \section{Learning Outcomes}

\begin{thebibliography}{9}
    \bibitem{angular} Angular. Retrieved from \url{https://angular.io/}.
    \bibitem{bootstrap} Bootstrap. Retrieved from
        \url{https://getbootstrap.com/}.
    \bibitem{cordova} Cordova. Retrieved from \url{https://cordova.apache.org/}.
    \bibitem{flask} Flask. Retrieved from \url{http://flask.pocoo.org/}.
    \bibitem{ionic} Ionic. Retrieved from \url{https://ionicframework.com/}.
    \bibitem{moore} Moore, Terry. \textit{Why X marks the unknown.} Retrieved
        from \url{https://cosmosmagazine.com/mathematics/why-x-marks-unknown-0}.
    \bibitem{nodejs} Node\@.js. Retrieved from \url{https://nodejs.org/en/}.
    \bibitem{npm} npm. Retrieved from \url{https://www.npmjs.com/}.
    \bibitem{pomax} Pomax. \textit{A Primer on Bézier Curves.} Retrieved from
        \url{https://pomax.github.io/bezierinfo/}.
    \bibitem{singularity} Singularity. Retrieved from \url{https://www.sylabs.io/singularity/}.
    \bibitem{typescript} TypeScript. Retrieved from
        \url{https://www.typescriptlang.org/}.
\end{thebibliography}

\end{document}
